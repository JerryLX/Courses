% !TeX spellcheck = en_US

%
% This is a borrowed LaTeX template file for lecture notes for CS267,
% Applications of Parallel Computing, UCBerkeley EECS Department.
% Now being used for CMU's 10725 Fall 2012 Optimization course
% taught by Geoff Gordon and Ryan Tibshirani.  When preparing 
% LaTeX notes for this class, please use this template.
%
% To familiarize yourself with this template, the body contains
% some examples of its use.  Look them over.  Then you can
% run LaTeX on this file.  After you have LaTeXed this file then
% you can look over the result either by printing it out with
% dvips or using xdvi. "pdflatex template.tex" should also work.
%

\documentclass[twoside]{article}
\setlength{\oddsidemargin}{0.25 in}
\setlength{\evensidemargin}{-0.25 in}
\setlength{\topmargin}{-0.6 in}
\setlength{\textwidth}{6.5 in}
\setlength{\textheight}{8.5 in}
\setlength{\headsep}{0.75 in}
\setlength{\parindent}{0 in}
\setlength{\parskip}{0.1 in}

%
% ADD PACKAGES here:
%
\usepackage{graphicx}
\usepackage{amsmath,amsfonts,graphicx}
\usepackage{algorithm}
\usepackage{algorithmic}
\usepackage{lipsum}
 \usepackage[margin=1in]{geometry} 
\usepackage{amsthm,amssymb}
\usepackage{bm}
\usepackage{enumerate}
\usepackage{booktabs}
%\renewcommand{\familydefault}{pag}
%\renewcommand{\familydefault}{pbk}
% https://en.wikibooks.org/wiki/LaTeX/Fonts

%
% The following commands set up the lecnum (lecture number)
% counter and make various numbering schemes work relative
% to the lecture number.
%
\newcounter{lecnum}
\renewcommand{\thepage}{\thelecnum-\arabic{page}}
\renewcommand{\thesection}{\thelecnum.\arabic{section}}
\renewcommand{\theequation}{\thelecnum.\arabic{equation}}
\renewcommand{\thefigure}{\thelecnum.\arabic{figure}}
\renewcommand{\thetable}{\thelecnum.\arabic{table}}

%
% The following macro is used to generate the header.
%
\newcommand{\lecture}[7]{
   \pagestyle{myheadings}
   \thispagestyle{plain}
   \newpage
   \setcounter{lecnum}{#1}
   \setcounter{page}{1}
   \noindent
   \begin{center}
   \framebox{
      \vbox{\vspace{2mm}
    \hbox to 6.28in { {\bf Computational Complexity 
	\hfill Autumn, 2018} }
       \vspace{4mm}
       \hbox to 6.28in { {\Large \hfill Solution : #2  \hfill} }
       \vspace{2mm}
       \hbox to 6.28in { {\it Lecturer: #3 \hfill Homework taker: #4} }
      \vspace{2mm}}
   }
   \end{center}
   \markboth{Solution : #2}{Solution : #2}
}

\newcommand{\N}{\mathbb{N}}
\newcommand{\Z}{\mathbb{Z}}

\newenvironment{problem}[2][Problem]{\begin{trivlist}
		\item[\hskip \labelsep {\bfseries #1}\hskip \labelsep {\bfseries #2.}]}{\end{trivlist}}

%
% Convention for citations is authors' initials followed by the year.
% For example, to cite a paper by Leighton and Maggs you would type
% \cite{LM89}, and to cite a paper by Strassen you would type \cite{S69}.
% (To avoid bibliography problems, for now we redefine the \cite command.)
% Also commands that create a suitable format for the reference list.
\renewcommand{\cite}[1]{[#1]}
\def\beginrefs{\begin{list}%
        {[\arabic{equation}]}{\usecounter{equation}
         \setlength{\leftmargin}{2.0truecm}\setlength{\labelsep}{0.4truecm}%
         \setlength{\labelwidth}{1.6truecm}}}
\def\endrefs{\end{list}}
\def\bibentry#1{\item[\hbox{[#1]}]}

%Use this command for a figure; it puts a figure in wherever you want it.
%usage: \fig{NUMBER}{SPACE-IN-INCHES}{CAPTION}
\newcommand{\fig}[3]{
			\vspace{#2}
			\begin{center}
			Figure \thelecnum.#1:~#3
			\end{center}
	}
% Use these for theorems, lemmas, proofs, etc.
\newtheorem{theorem}{Theorem}[lecnum]
\newtheorem{lemma}[theorem]{Lemma}
\newtheorem{proposition}[theorem]{Proposition}
\newtheorem{claim}[theorem]{Claim}
\newtheorem{corollary}[theorem]{Corollary}
\newtheorem{definition}[theorem]{Definition}
\newtheorem{remark}[theorem]{Remark}
%\newenvironment{proof}{{\bf Proof:}}{\hfill\rule{2mm}{2mm}}
\newenvironment{solution}{{\bf Solution:}}{\hfill\rule{2mm}{2mm}}

% **** IF YOU WANT TO DEFINE ADDITIONAL MACROS FOR YOURSELF, PUT THEM HERE:

\newcommand\E{\mathbb{E}}

\begin{document}
%FILL IN THE RIGHT INFO.
%\lecture{**LECTURE-NUMBER**}{**DATE**}{**LECTURER**}{**SCRIBE**}
\lecture{1}{Homework 2}{Yuxi Fu}{Xu Li 018033210002}



 % \textbf{Notification:} You can take 5 problems randomly from all of 15 problems except the problem you design.
  %\textbf{Due Time:}March 12

\begin{problem}{2.3}
\end{problem}
\begin{solution}
Since each ration number can be represented by 2 integer. The matrix $A$ can be represented using $O(mn)$ bits. Given a pair $<A,b>$, we can calculate all $x$ satisfying $Ax=b$ in $O(n^3)$ time. And it takes $O(n)$ time to verify $x$, i.e., it takes $<A,b>$ can be verified in polynomial time and therefore LINEQ is in NP.
\end{solution}

\begin{problem}{2.6}
\end{problem} 
\begin{solution}
\begin{enumerate}[(a)]
	\item We construct the NDTM $NU$ as follows.
Similar as Universal Turing Machine, given a NDTM $M_\alpha$, we use one work tape to simulate all work tapes of $M_\alpha$. Since UTM can simulate a TM in $CT\log T$ steps, $NU$ can also simulate a NDTM in $CT\log T$ steps.
	\item We construct the NDTM $NU$ as follows. It guesses a sequence of snapshots and a sequence of choices by running the input machine on the input value without looking at the worktapes.
	It then verifies for each worktape of the input machine if the snapshots are legal.
	I To follow the content change of the tape being verified, V needs an additional tape. The length of a snapshot is a constant only relies on $M_\alpha$. The verification takes the former snapshot for input and can be down also in constant time. Therefore $NU$ runs for at most
	$CT$ steps.
\end{enumerate}
\end{solution}

\begin{problem}{2.13}
\end{problem}
\begin{solution}
\begin{enumerate}[(a)]
	\item Modify the machine $M$ so that it clears up its work tape before outputting a 1 and
	moves both heads to one. Then the final snapshot and head locations are unique. 
	The proof of Cook-Levin Reduction gives a one-to-one and onto mapping between the set of certificates and the set of satisfying assignments. So the number of satisfying truth assignment equals to that of $M$ accepting computation paths.
 	\item To reduct SAT to 3SAT, we transform the CNF in SAT to clauses in 3SAT form as follows. We assume the CNF is $E=e_1 \vee e_2 \vee ... \vee e_k$ where each $e_i$	is a disjunction of literals. 
 	\begin{itemize}
 		\item If $e_i$ is a single literal, say $x$, we introduce two new variables $u$ and $v$. We replace $x$ by  the conjunction of four clauses as $(x\vee u\vee v)\wedge(x\vee u\vee \overline{v})\wedge(x\vee \overline{u}\vee v)\wedge(x\vee \overline{u}\vee \overline{v})$.
 		Since $u$ and $v$ appear in all combinations, the only way to satisfy all four clauses is to
 		make $x$ true.
 		\item Suppose an $e_i$ 	is the disjunction of two literals, $(x \vee y)$. We introduce a new variable $z$ and replace $e_i$ by the conjunction of two clauses $(x \vee y \vee \overline{z})\wedge(x \vee y \vee z)$. As in case 1, the
 		only way to satisfy both clauses is to satisfy $(x \vee y)$.
 		\item If an $e_i$ is the disjunction of three literals it is already in the form required for 3-CNF, so we take $e_i$.
 		\item 
Suppose $e_i = (x_1\vee x_2\vee...\vee x_m)$ for some $m \ge 4$. We introduce new variables $y_1, y_2, ..., y_{m−3}$ 	and replace $e_i$ by the conjunction of clauses
 		\begin{equation}
 		(x_1 \vee x_2 \vee y_1)\wedge(x_3 \vee \overline{y_1} \vee y_2)\wedge(x_4 \vee \overline{y_2} \vee y_3)...(x_{m−2} \vee \overline{y_{m-4}} \vee y_{m-3})\wedge(x_{m-1} \vee x_m \vee y_{m-3}). 
 		\end{equation}
 	\end{itemize}
 \end{enumerate}
\end{solution}

\begin{problem}{4.10}
\end{problem}

\begin{solution}
With perfect information of the game, we can construct the configuration graph. 
We can walk through the graph to find if these is a series of moves for player A that satisfies no matter what move that player B make, there is still a path leading to 'Player A wins'.
If such move for  player A exists, there is a winning strategy for player A (choose).
Otherwise, there is a series of moves for player B satisfying no matter what moves took by player A, player B can also find a way to win.
\end{solution}

\begin{problem}{4.11}
\end{problem}
\begin{solution}
 Let $L\in coNSPACE(s(n))$. Then there is a non-deterministic machine $M$ using space $s(n)$ and with the following property: if $x\in L$ then $M(x)$ accepts on every computation path, while if $x \notin L$ then there is some computation path on which $M(x)$ rejects. Considering the configuration graph $G_M$ of $M(x)$ for some input $x$ of length $n$, we see that $x \in L$ iff there is no directed path in $G_M$ from the starting configuration to the rejecting configuration. Since $G_M$ has $V = 2^{O(s(n))}$ vertices, and the existence of an edge between two vertices $i$ and $j$ can be determined in $O(s(n)) = O(log V )$ space, we can decide $L$ in space $O(log V ) = O(s(n))$.
\end{solution}
\end{document}