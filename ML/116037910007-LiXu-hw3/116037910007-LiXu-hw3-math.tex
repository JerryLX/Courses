% !TeX spellcheck = en_US

%
% This is a borrowed LaTeX template file for lecture notes for CS267,
% Applications of Parallel Computing, UCBerkeley EECS Department.
% Now being used for CMU's 10725 Fall 2012 Optimization course
% taught by Geoff Gordon and Ryan Tibshirani.  When preparing 
% LaTeX notes for this class, please use this template.
%
% To familiarize yourself with this template, the body contains
% some examples of its use.  Look them over.  Then you can
% run LaTeX on this file.  After you have LaTeXed this file then
% you can look over the result either by printing it out with
% dvips or using xdvi. "pdflatex template.tex" should also work.
%

\documentclass[twoside]{article}
\setlength{\oddsidemargin}{0.25 in}
\setlength{\evensidemargin}{-0.25 in}
\setlength{\topmargin}{-0.6 in}
\setlength{\textwidth}{6.5 in}
\setlength{\textheight}{8.5 in}
\setlength{\headsep}{0.75 in}
\setlength{\parindent}{0 in}
\setlength{\parskip}{0.1 in}

%
% ADD PACKAGES here:
%
\usepackage{graphicx}
\usepackage{amsmath,amsfonts,graphicx}
\usepackage{algorithm}
\usepackage{algorithmic}
\usepackage{lipsum}
 \usepackage[margin=1in]{geometry} 
\usepackage{amsthm,amssymb}
\usepackage{bm}

\usepackage{booktabs}
%\renewcommand{\familydefault}{pag}
%\renewcommand{\familydefault}{pbk}
% https://en.wikibooks.org/wiki/LaTeX/Fonts

%
% The following commands set up the lecnum (lecture number)
% counter and make various numbering schemes work relative
% to the lecture number.
%
\newcounter{lecnum}
\renewcommand{\thepage}{\thelecnum-\arabic{page}}
\renewcommand{\thesection}{\thelecnum.\arabic{section}}
\renewcommand{\theequation}{\thelecnum.\arabic{equation}}
\renewcommand{\thefigure}{\thelecnum.\arabic{figure}}
\renewcommand{\thetable}{\thelecnum.\arabic{table}}

%
% The following macro is used to generate the header.
%
\newcommand{\lecture}[7]{
   \pagestyle{myheadings}
   \thispagestyle{plain}
   \newpage
   \setcounter{lecnum}{#1}
   \setcounter{page}{1}
   \noindent
   \begin{center}
   \framebox{
      \vbox{\vspace{2mm}
    \hbox to 6.28in { {\bf Machine Learning
	\hfill Spring 2017} }
       \vspace{4mm}
       \hbox to 6.28in { {\Large \hfill Solution : #2  \hfill} }
       \vspace{2mm}
       \hbox to 6.28in { {\it Lecturer: #3 \hfill Homework taker: #4} }
      \vspace{2mm}}
   }
   \end{center}
   \markboth{Solution : #2}{Solution : #2}
}

\newcommand{\N}{\mathbb{N}}
\newcommand{\Z}{\mathbb{Z}}

\newenvironment{problem}[2][Problem]{\begin{trivlist}
		\item[\hskip \labelsep {\bfseries #1}\hskip \labelsep {\bfseries #2.}]}{\end{trivlist}}

%
% Convention for citations is authors' initials followed by the year.
% For example, to cite a paper by Leighton and Maggs you would type
% \cite{LM89}, and to cite a paper by Strassen you would type \cite{S69}.
% (To avoid bibliography problems, for now we redefine the \cite command.)
% Also commands that create a suitable format for the reference list.
\renewcommand{\cite}[1]{[#1]}
\def\beginrefs{\begin{list}%
        {[\arabic{equation}]}{\usecounter{equation}
         \setlength{\leftmargin}{2.0truecm}\setlength{\labelsep}{0.4truecm}%
         \setlength{\labelwidth}{1.6truecm}}}
\def\endrefs{\end{list}}
\def\bibentry#1{\item[\hbox{[#1]}]}

%Use this command for a figure; it puts a figure in wherever you want it.
%usage: \fig{NUMBER}{SPACE-IN-INCHES}{CAPTION}
\newcommand{\fig}[3]{
			\vspace{#2}
			\begin{center}
			Figure \thelecnum.#1:~#3
			\end{center}
	}
% Use these for theorems, lemmas, proofs, etc.
\newtheorem{theorem}{Theorem}[lecnum]
\newtheorem{lemma}[theorem]{Lemma}
\newtheorem{proposition}[theorem]{Proposition}
\newtheorem{claim}[theorem]{Claim}
\newtheorem{corollary}[theorem]{Corollary}
\newtheorem{definition}[theorem]{Definition}
\newtheorem{remark}[theorem]{Remark}
%\newenvironment{proof}{{\bf Proof:}}{\hfill\rule{2mm}{2mm}}
\newenvironment{solution}{{\bf Solution:}}{\hfill\rule{2mm}{2mm}}

% **** IF YOU WANT TO DEFINE ADDITIONAL MACROS FOR YOURSELF, PUT THEM HERE:

\newcommand\E{\mathbb{E}}

\begin{document}
%FILL IN THE RIGHT INFO.
%\lecture{**LECTURE-NUMBER**}{**DATE**}{**LECTURER**}{**SCRIBE**}
\lecture{3}{Homework 3}{Yang Yang}{Li Xu}



 % \textbf{Notification:} You can take 5 problems randomly from all of 15 problems except the problem you design.
  \textbf{Due Time:}May 19

\begin{problem}{1}
\end{problem}

\begin{solution}

\emph{Answer for problem (1):}
$$
\frac{\partial L}{\partial b_1} = \frac{\partial L}{\partial z_m}\sigma'(a_1)\prod_{k=2}^{m}\sigma'(a_k)w_k
$$

\emph{Answer for problem (2):}

(a)
For sigmoid function $\sigma$, we have:
$$\sigma'(x) = \sigma(x)(1-\sigma(x))\le \frac{1}{4}$$
Thus:
$$
|w_j\sigma(a_j)|<\frac{1}{4}<1
$$
So when $m$ is large, $\prod_{m}^{k=2}\sigma'(a_k)w_k$ tends to be 0.

(b)
Even if we have a large $w$, $\sigma'(a)$ is still very small as 
$$\sigma'(a) = \sigma(a)(1-\sigma(a))$$ 

\emph{Answer for problem (3):}

As long as $a> 0$, $\sigma'(a) = 1$. Thus there is no vanishing problem.
\end{solution}

\begin{problem}{2}
\end{problem} 
\begin{solution}
	
	We first prove $VC(h)\ge n+1$, that is to say, we first find a set of size $n+1$ that h can shatter. It's easy to construct. For example, we let the set equals to
	$$ 
	{(0,0,...,0)^T,(1,0,...,0)^T,...,(n,0,...,0)^T}
	$$
	and $(0,0,...,0)$ is in class 1 the others are in class 2. If we let $a=(1,0,...,0)^T$ and $b=0$, $h(x)$ then can shatter this set.
	
	Next we prove $VC(h)\le n+1$. There is no set with size $n+2$ that h can shatter. For a set of n + 2 	points in $\mathbb{R}^n$ can be partitioned into two disjoint subsets $S_1$ and $S_2$ such that their convex hulls intersect. Thus it cannot be shattered linearly.
	
\end{solution}

\begin{problem}{3}
\end{problem} 
\begin{solution}
	In this data set, Relief is used for feature chosen. I chose top five features for each pair. For D and O the chosen features are [6,8,9,10,13], and for O and X are [8,9,10,11,13].
	
	The number of nodes in hidden layer is set to 6 and the number of iterations is set to 200. learning rate is set to 0.005. Sigmoid function is used both in hidden layer and output layer. Other values are initialed randomly between 0 to 0.2.
	
	The final accuracy rate between D and O is 98.93\% and 98.92\%between O and X.
\end{solution}
\end{document}