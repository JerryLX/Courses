% !TeX spellcheck = en_US

%
% This is a borrowed LaTeX template file for lecture notes for CS267,
% Applications of Parallel Computing, UCBerkeley EECS Department.
% Now being used for CMU's 10725 Fall 2012 Optimization course
% taught by Geoff Gordon and Ryan Tibshirani.  When preparing 
% LaTeX notes for this class, please use this template.
%
% To familiarize yourself with this template, the body contains
% some examples of its use.  Look them over.  Then you can
% run LaTeX on this file.  After you have LaTeXed this file then
% you can look over the result either by printing it out with
% dvips or using xdvi. "pdflatex template.tex" should also work.
%

\documentclass[twoside]{article}
\setlength{\oddsidemargin}{0.25 in}
\setlength{\evensidemargin}{-0.25 in}
\setlength{\topmargin}{-0.6 in}
\setlength{\textwidth}{6.5 in}
\setlength{\textheight}{8.5 in}
\setlength{\headsep}{0.75 in}
\setlength{\parindent}{0 in}
\setlength{\parskip}{0.1 in}

%
% ADD PACKAGES here:
%
\usepackage{graphicx}
\usepackage{amsmath,amsfonts,graphicx}
\usepackage{algorithm}
\usepackage{algorithmic}
\usepackage{lipsum}
 \usepackage[margin=1in]{geometry} 
\usepackage{amsthm,amssymb}
\usepackage{bm}

\usepackage{booktabs}
%\renewcommand{\familydefault}{pag}
%\renewcommand{\familydefault}{pbk}
% https://en.wikibooks.org/wiki/LaTeX/Fonts

%
% The following commands set up the lecnum (lecture number)
% counter and make various numbering schemes work relative
% to the lecture number.
%
\newcounter{lecnum}
\renewcommand{\thepage}{\thelecnum-\arabic{page}}
\renewcommand{\thesection}{\thelecnum.\arabic{section}}
\renewcommand{\theequation}{\thelecnum.\arabic{equation}}
\renewcommand{\thefigure}{\thelecnum.\arabic{figure}}
\renewcommand{\thetable}{\thelecnum.\arabic{table}}

%
% The following macro is used to generate the header.
%
\newcommand{\lecture}[7]{
   \pagestyle{myheadings}
   \thispagestyle{plain}
   \newpage
   \setcounter{lecnum}{#1}
   \setcounter{page}{1}
   \noindent
   \begin{center}
   \framebox{
      \vbox{\vspace{2mm}
    \hbox to 6.28in { {\bf Machine Learning
	\hfill Spring 2017} }
       \vspace{4mm}
       \hbox to 6.28in { {\Large \hfill Solution : #2  \hfill} }
       \vspace{2mm}
       \hbox to 6.28in { {\it Lecturer: #3 \hfill Homework taker: #4} }
      \vspace{2mm}}
   }
   \end{center}
   \markboth{Solution : #2}{Solution : #2}
}

\newcommand{\N}{\mathbb{N}}
\newcommand{\Z}{\mathbb{Z}}

\newenvironment{problem}[2][Problem]{\begin{trivlist}
		\item[\hskip \labelsep {\bfseries #1}\hskip \labelsep {\bfseries #2.}]}{\end{trivlist}}

%
% Convention for citations is authors' initials followed by the year.
% For example, to cite a paper by Leighton and Maggs you would type
% \cite{LM89}, and to cite a paper by Strassen you would type \cite{S69}.
% (To avoid bibliography problems, for now we redefine the \cite command.)
% Also commands that create a suitable format for the reference list.
\renewcommand{\cite}[1]{[#1]}
\def\beginrefs{\begin{list}%
        {[\arabic{equation}]}{\usecounter{equation}
         \setlength{\leftmargin}{2.0truecm}\setlength{\labelsep}{0.4truecm}%
         \setlength{\labelwidth}{1.6truecm}}}
\def\endrefs{\end{list}}
\def\bibentry#1{\item[\hbox{[#1]}]}

%Use this command for a figure; it puts a figure in wherever you want it.
%usage: \fig{NUMBER}{SPACE-IN-INCHES}{CAPTION}
\newcommand{\fig}[3]{
			\vspace{#2}
			\begin{center}
			Figure \thelecnum.#1:~#3
			\end{center}
	}
% Use these for theorems, lemmas, proofs, etc.
\newtheorem{theorem}{Theorem}[lecnum]
\newtheorem{lemma}[theorem]{Lemma}
\newtheorem{proposition}[theorem]{Proposition}
\newtheorem{claim}[theorem]{Claim}
\newtheorem{corollary}[theorem]{Corollary}
\newtheorem{definition}[theorem]{Definition}
\newtheorem{remark}[theorem]{Remark}
%\newenvironment{proof}{{\bf Proof:}}{\hfill\rule{2mm}{2mm}}
\newenvironment{solution}{{\bf Solution:}}{\hfill\rule{2mm}{2mm}}

% **** IF YOU WANT TO DEFINE ADDITIONAL MACROS FOR YOURSELF, PUT THEM HERE:

\newcommand\E{\mathbb{E}}

\begin{document}
%FILL IN THE RIGHT INFO.
%\lecture{**LECTURE-NUMBER**}{**DATE**}{**LECTURER**}{**SCRIBE**}
\lecture{1}{Homework 1}{Yang Yang}{Li Xu}



 % \textbf{Notification:} You can take 5 problems randomly from all of 15 problems except the problem you design.
  \textbf{Due Time:}March 12

\begin{problem}{1}
	For parameter w, try to prove that logistic regression function $y=\frac{1}{1+e^{-(\bm{w^Tx}+b)}}$ is non-convex, but its logarithmic likelihood function $l(\bm{w})=\sum_{i=1}^m(-y_i(\bm{w^Tx_i}+b)+\ln(1+e^{\bm{w^Tx_i}+b}))$ is convex.
\end{problem}

\begin{solution}
	To prove $y=\frac{1}{1+e^{-(\bm{w^Tx}+b)}}$ is non-convex, we just need to prove $g(z)=\frac{1}{1+e^{-z}}$ is non-convex as $z=\bm{w^Tx}+b$ is linear.
	
	\begin{proof}
		$g(z)=\frac{1}{1+e^{-z}}$ is non-convex
	\end{proof}
	
	\begin{align*}
		g'(z) &= \frac{d}{dz}\frac{1}{1+e^{-z}} \\
			  &= \frac{1}{1+e^{-z}}(1-\frac{1}{1+e^{-z}}) \\
			  &= g(z)(1-g(z))
	\end{align*}
	\begin{align*}
	g''(z) &= \frac{d}{dz}g(z) \\
		&= g'(z)(1-g(z))-g(z)g'(z) \\
		&= g'(z) - 2g(z)g'(z) \\
		&= g(z)(1-g(z))(1-2g(z))
	\end{align*}
	As the range of g(z) is from $(0,+\infty)$, g''(z)  is not constant greater than 0. So $g(z)=\frac{1}{1+e^{-z}}$ is non-convex. 
	
	Similarly, to prove $l(\bm{w})=\sum_{i=1}^m(-y_i(\bm{w^Tx_i}+b)+\ln(1+e^{\bm{w^Tx_i}+b}))$ is convex, we just need to prove $\ln(1+e^{z})$ is convex as both $-y_i(\bm{w^Tx_i}+b$ and $\bm{w^Tx_i}+b$ is linear to $\bm{w}$.
	
	\begin{proof}
		$g(z) = \ln(1+e^{z})$ is convex
	\end{proof}
	$$
	g'(z) = \frac{d}{dz}\ln(1+e^{z}) 
		= \frac{e^{z}}{1+e^{z}}
	$$
	
	$$	g''(z) = \frac{d}{dz}\frac{e^{z}}{1+e^{z}} 
		= \frac{e^z(1+e^z) - e^{2z}}{(1+e^z)^2} 
		= \frac{e^z}{(1+e^z)^2} >0
	$$
\end{solution}

\begin{problem}{2}
	Using the technique of Lagrange multipliers, show that minimization of the regularized error function\\
	\begin{align*}
	\frac{1}{2}\sum_{n=1}^N\{t_n-\bm{w^T\phi}(x_n)\}^2+\frac{\lambda}{2}\sum_{j=1}^M|w_j|^q
	\end{align*}
	is equivalent to minimizing the unregularized sum-of-squares error\\
	\begin{align*}
	E_D(\bm{w})=\frac{1}{2}\sum_{n=1}^N\{t_n-\bm{w^T\phi}(x_n)\}^2
	\end{align*}
	subject to the constraint $\sum_{j=1}^M|w_j|^q\le \eta$.Discuss the relationship between the parameters $\eta$ and $\lambda$.
\end{problem} 
\begin{solution}
	We first define $E(\bm{w})=	\frac{1}{2}\sum_{n=1}^N\{t_n-\bm{w^T\phi}(x_n)\}^2+\frac{\lambda}{2}\sum_{j=1}^M|w_j|^q$
	
	To minimize $E(\bm{w})$, we have
	$$
	\frac{\partial E}{\partial \bm{w}} = 0
	$$
	
	To minimize $E_D(\bm{w})$ subject to $\sum_{j=1}^M|w_j|^q\le \eta$. We define:
	$$
	g(\bm{w}) = \frac{1}{2}(\sum_{j=1}^M|w_j|^q - \eta)
	$$
	Using Lagrange multipliers, let 
	$$L(\bm{w}, \lambda) = E_D(\bm{w})+\lambda \cdot g(\bm{w})$$
	and solve
	$$
	\nabla_{\bm{w}}L(\bm{w}, \lambda) = 0
	$$
	with $\lambda \cdot g(\bm{w}) = 0$.
	
	As $L(\bm{w}, \lambda) = E(\bm{w} - \lambda \eta)$ and $\lambda \eta$ is not relevant to $\bm{w}$.
	
	$$\nabla_{\bm{w}}L(\bm{w}, \lambda) = 0$$ 
	is equivalent to 
	$$\frac{\partial E}{\partial \bm{w}} = 0$$
	So minimize $E(\bm{w})$ is equivalent to minimize $E_D(\bm{w})$ subject to $\sum_{j=1}^M|w_j|^q\le \eta$.
	
	For the relationship between $\lambda$ and $\eta$. As we have $\lambda \cdot g(\bm{w}) = 0$, if $\lambda \ne 0$, $\eta = \sum_{j=1}^M|w_j|^q$.
\end{solution}

\begin{problem}{3}
	Consider a data set in which each data point $t_n$ is associated with a weighting factor $r_n > 0$, so that the sum-of-squares error function becomes $$E_D(\mathbf{w}) = \frac{1}{2} \sum_{n=1}^N r_n \{t_n - \mathbf{w}^T\phi(\mathbf{x}_n)\}^2.$$ where $\phi(\mathbf{x}_n)$ is basis function.  Find an expression for the solution $\mathbf{w}^*$ that minimizes this error function. 
\end{problem}
\begin{solution}
	To minimize $$E_D(\mathbf{w}) = \frac{1}{2} \sum_{n=1}^N r_n \{t_n - \mathbf{w}^T\phi(\mathbf{x}_n)\}^2.$$
	
	$\mathbf{w}$ should satisfy:
	$$
	\frac{\partial}{\partial \mathbf{w}} E_D(\mathbf{w}) = - \sum_{n=1}^N r_n \{t_n - \mathbf{w}^T\phi(\mathbf{x}_n)\}\phi(\mathbf{x}_n) = 0
	$$
	Solve for $\mathbf{w}$:
	\begin{align*}
	\sum_{n=1}^N r_n t_n \phi(\mathbf{x}_n) &= (\sum_{n=1}^N r_n \phi(\mathbf{x}_n) \phi(\mathbf{x}_n)^T)\mathbf{w} \\
				\mathbf{w} &= (\sum_{n=1}^N r_n \phi(\mathbf{x}_n) \phi(\mathbf{x}_n)^T)^{-1} (\sum_{n=1}^N r_n t_n \phi(\mathbf{x}_n))
	\end{align*}

\end{solution}

\begin{problem}{4}
	......
\end{problem}

\begin{solution}
We define $\mu_{ic} = P(y_i = c | x_i, W), y_{ic} =1 \{y_i = c\}$

(a)
$$
l(W) = log \prod_{i=1}^{n} \prod_{c=1}^{C} y_{ic}log \mu_{ic} = \sum_{i=1}^{n}(\sum_{c=1}^{C}y_{ic}w_c^T x_i - log\sum_{c'=1}^{C}exp(w_{c'}^T x_i))
$$

(b)
\begin{align*}
g_c(W) &= \frac{\partial}{\partial w_c}\sum_{i=1}^{n}(\sum_{c=1}^{C} y_{ic}w_c^Tx_i - log\sum_{c'=1}^{C} exp(w_{c'}^T x_i)) \\
&= \sum_{i=1}^{n}(\frac{\partial}{\partial w_c}\sum_{c=1}^{C}y_{ic}w_c^Tx_i - \frac{\partial}{\partial w_c} log\sum_{c'=1}^{C}exp(w_{c'}^Tx_i)) \\
&= \sum_{i=1}^{n}(y_{ic}x_i - \frac{\frac{\partial}{\partial w_c} \sum_{c'=1}^{C}exp(w_{c'}^T x_i)}{\sum_{c'=1}^{C} exp(w_{c'}^Tx_i)}) \\
&=  \sum_{i=1}^{n}(y_{ic}x_i - \frac{exp(w_{c}^Tx_i)x_i}{\sum_{c'=1}^{C} exp(w_{c'}^Tx_i)}) \\
&= \sum_{i=1}^{n}(y_{ic}-\mu_{ic})x_i
\end{align*}

(c) $\delta_{cc'}$ denotes the Dirac delta function and is equal to one if $c' = c$ and zero otherwise.

\begin{align*}
H_{c,c'}(W) &= \frac{\partial}{\partial w_c}g_{c'}(W) \\
&= \frac{\partial}{\partial w_c} \sum_{i=1}^{n}(y_{ic'}x_i - \frac{exp(w_{c'}^Tx_i)x_i}{\sum_{c''=1}^{C} exp(w_{c''}^T x_i)})\\
&= - \sum_{i=1}^{n}\frac{\partial}{\partial w_c} \frac{exp(w_{c'}^T x_i)x_i}{\sum_{c''=1}^{C} exp(w_{c''}^Tx_i)} \\
&= - \sum_{i=1}^{n}(\delta_{cc'}\mu_{ic'}\mu_{ic})x_i x_i^T\\
&=  \sum_{i=1}^{n}\mu_{ic}(\mu_{ic'}-\delta_{cc'})x_i x_i^T
\end{align*}
	
\end{solution}

\begin{problem}{5}
	......
\end{problem}

\begin{solution}
	program is in the folder "B16037910007-LiXu-hw1-program"
	using 1. Stochastic Gradient Descent 2. Batch Gradient Ascent Method 3. Newton’s method 4. Normal Equation
	
	comparing RMSE: Normal Equation has best result.
\end{solution}
\end{document}