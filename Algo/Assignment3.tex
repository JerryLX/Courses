% !TeX spellcheck = en_US

%
% This is a borrowed LaTeX template file for lecture notes for CS267,
% Applications of Parallel Computing, UCBerkeley EECS Department.
% Now being used for CMU's 10725 Fall 2012 Optimization course
% taught by Geoff Gordon and Ryan Tibshirani.  When preparing 
% LaTeX notes for this class, please use this template.
%
% To familiarize yourself with this template, the body contains
% some examples of its use.  Look them over.  Then you can
% run LaTeX on this file.  After you have LaTeXed this file then
% you can look over the result either by printing it out with
% dvips or using xdvi. "pdflatex template.tex" should also work.
%

\documentclass[twoside]{article}
\setlength{\oddsidemargin}{0.25 in}
\setlength{\evensidemargin}{-0.25 in}
\setlength{\topmargin}{-0.6 in}
\setlength{\textwidth}{6.5 in}
\setlength{\textheight}{8.5 in}
\setlength{\headsep}{0.75 in}
\setlength{\parindent}{0 in}
\setlength{\parskip}{0.1 in}

%
% ADD PACKAGES here:
%
\usepackage{graphicx}
\usepackage{amsmath,amsfonts,graphicx}
\usepackage{algorithm}
\usepackage{algorithmic}
\usepackage{lipsum}
 \usepackage[margin=1in]{geometry} 
\usepackage{amsthm,amssymb}
\usepackage{bm}

\usepackage{booktabs}
%\renewcommand{\familydefault}{pag}
%\renewcommand{\familydefault}{pbk}
% https://en.wikibooks.org/wiki/LaTeX/Fonts

%
% The following commands set up the lecnum (lecture number)
% counter and make various numbering schemes work relative
% to the lecture number.
%
\newcounter{lecnum}
\renewcommand{\thepage}{\thelecnum-\arabic{page}}
\renewcommand{\thesection}{\thelecnum.\arabic{section}}
\renewcommand{\theequation}{\thelecnum.\arabic{equation}}
\renewcommand{\thefigure}{\thelecnum.\arabic{figure}}
\renewcommand{\thetable}{\thelecnum.\arabic{table}}

%
% The following macro is used to generate the header.
%
\newcommand{\lecture}[7]{
   \pagestyle{myheadings}
   \thispagestyle{plain}
   \newpage
   \setcounter{lecnum}{#1}
   \setcounter{page}{1}
   \noindent
   \begin{center}
   \framebox{
      \vbox{\vspace{2mm}
    \hbox to 6.28in { {\bf Algorithm
	\hfill Spring 2017} }
       \vspace{4mm}
       \hbox to 6.28in { {\Large \hfill Solution : #2  \hfill} }
       \vspace{2mm}
       \hbox to 6.28in { {\it Lecturer: #3 \hfill Homework taker: #4} }
      \vspace{2mm}}
   }
   \end{center}
   \markboth{Solution : #2}{Solution : #2}
}

\newcommand{\N}{\mathbb{N}}
\newcommand{\Z}{\mathbb{Z}}

\newenvironment{problem}[2][Problem]{\begin{trivlist}
		\item[\hskip \labelsep {\bfseries #1}\hskip \labelsep {\bfseries #2.}]}{\end{trivlist}}

%
% Convention for citations is authors' initials followed by the year.
% For example, to cite a paper by Leighton and Maggs you would type
% \cite{LM89}, and to cite a paper by Strassen you would type \cite{S69}.
% (To avoid bibliography problems, for now we redefine the \cite command.)
% Also commands that create a suitable format for the reference list.
\renewcommand{\cite}[1]{[#1]}
\def\beginrefs{\begin{list}%
        {[\arabic{equation}]}{\usecounter{equation}
         \setlength{\leftmargin}{2.0truecm}\setlength{\labelsep}{0.4truecm}%
         \setlength{\labelwidth}{1.6truecm}}}
\def\endrefs{\end{list}}
\def\bibentry#1{\item[\hbox{[#1]}]}

%Use this command for a figure; it puts a figure in wherever you want it.
%usage: \fig{NUMBER}{SPACE-IN-INCHES}{CAPTION}
\newcommand{\fig}[3]{
			\vspace{#2}
			\begin{center}
			Figure \thelecnum.#1:~#3
			\end{center}
	}
% Use these for theorems, lemmas, proofs, etc.
\newtheorem{theorem}{Theorem}[lecnum]
\newtheorem{lemma}[theorem]{Lemma}
\newtheorem{proposition}[theorem]{Proposition}
\newtheorem{claim}[theorem]{Claim}
\newtheorem{corollary}[theorem]{Corollary}
\newtheorem{definition}[theorem]{Definition}
\newtheorem{remark}[theorem]{Remark}
%\newenvironment{proof}{{\bf Proof:}}{\hfill\rule{2mm}{2mm}}
\newenvironment{solution}{{\bf Solution:}}{\hfill\rule{2mm}{2mm}}

% **** IF YOU WANT TO DEFINE ADDITIONAL MACROS FOR YOURSELF, PUT THEM HERE:

\newcommand\E{\mathbb{E}}

\begin{document}
%FILL IN THE RIGHT INFO.
%\lecture{**LECTURE-NUMBER**}{**DATE**}{**LECTURER**}{**SCRIBE**}
\lecture{2}{Assignment 3}{Li Guoqiang}{Yu Han}



 % \textbf{Notification:} You can take 5 problems randomly from all of 15 problems except the problem you design.
  \textbf{Due Time:}May 11

\begin{problem}{1}
	Prove the Konig theorem: Let $G$ be bipartite, then cardinality of maximum matching = cardinality of minimum vertex cover.
\end{problem}

\begin{solution}
	
	We use $C_{match}$ donates the cardinality of maximum matching and  $C_{cover}$ donates the cardinality of minimum vertex cover.
		
	$M$ is a maximum matching for $G$. Then no vertex in a vertex cover can cover more than one edge of M, for M is a set of edges no two of which share an endpoint. Thus:
	$$
	C_{cover} \ge |M| = C_{match}
	$$
	
	Let vertex set $V$ be partitioned into left set $L$ and right set $R$.
	let $U$ be the set of unmatched vertices in $L$ (possibly empty), and let $Z$ be the set of vertices that are either in $U$ or are connected to $U$ by alternating paths (paths that alternate between edges that are in the matching and edges that are not in the matching). Let 
	$$
	K = (L \setminus Z) \cup (R \cap Z)
	$$
	
	$K$ forms a vertex cover and every vertex in $K$ is an endpoint of a matched edge. Thus:
	$$
	C_{cover} \le C_{match}
	$$
	
	Thus:
	$$
	C_{cover} = C_{match}
	$$
\end{solution}

\begin{problem}{2}
	Use layering to get a factor $f$ approximation algorithm for set cover, where $f$ is the frequency of the most frequent element. Provide a tight example for this algorithm.
\end{problem} 
\begin{solution}

1.$\mathcal{S}_0 = \mathcal{S}, C = \emptyset, i=0$

2.Remove $S$ where $S \cap U = \emptyset$, say this set is $\mathcal{S}_i$

3.Compute $c = min\{w(S)/|S|\}$ for all $S \in \mathcal{S}_i$

4.Let $t_i(S) = c \cdot |S|$ and $w(S) = w(S) - t_i(S)$ for all $S \in \mathcal{S}_i$

5.Let $W_i = \{S \in \mathcal{S}_i | w(S) = 0\}, C = C \cup W_i$

6.Let $U=U/\bigcup_{S\in W_i}$, Increase $i$ by 1 and goto step 2 until $U$ is empty set.
	
	
	Tight Example:
	$S_1 = S_2 = \cdots = S_m = U$. The optimistic solution will choose only one subset, and the approximate solution will choose all the subsets.
\end{solution}

\begin{problem}{3}
	Given an undirected graph. The problem is to remove a minimum number of edges such
	that the residual graph contains no triangle. (I.e., there is no three vertices a, b, c such that edges
	(a, b),(b, c),(c, a) are all in the residual graph.) Give a factor 3 approximation algorithm.
\end{problem}

\begin{solution}

Repeatedly search the graph $G$. If there is a triangle (u,v,w), remove edge (u,v),(v,w) and (u,w) from $G$.

\end{solution}

\begin{problem}{4}
	Given n points in $\mathbb{R}^2$
	, define the optimal Euclidean Steiner tree to be a minimum length tree
	containing all n points and any other subset of points from $\mathbb{R}^2$
	. Prove that each of the additional
	points must have degree three, with all three angles being 120◦	.

\end{problem} 
\begin{solution}
	
	If a additional point has degree one, this additional point is obviously redundant.
	
	If a additional point has degree two,  this additional point is also redundant for	triangle inequality.
	
	If a additional point has degree more than two, it should be Fermat points or else we always could construct a smaller Steiner tree.

\end{solution}

\begin{problem}{5}
Consider a more restricted algorithm than First-Fit, called Next-Fit, which tries to pack the next
item only in the most recently started bin. If it does not fit, it is packed in a new bin. Show that this
algorithm also achieves factor 2. Give a factor 2 tight example.
\end{problem} 

\begin{solution}
	
	Suppose we have a set $I = \{1,2,...,n\}$ of items and each item is of size $s(i) \in (0,1]$. Let $k$ be the number of bins found by NEXT-Fit and the corresponding assignment is denoted by $A:I\rightarrow\{1,2,...k\}$. Let $k^\star$ be the optimal solution.
	
	$$
	k^\star \ge \lceil \sum_{i\in I}s(i)\rceil
	$$
	For bins $j = 1,2,...,\lfloor k/2 \rfloor$, we have
	$$
	\sum_{i:A(i)\in \{2j-1,2j\}}s(i) > 1
	$$
	Combining inequalities:
	$$
	\lfloor \frac{k}{2} \rfloor < \sum_{i\in I}s(i)
	$$
	Thus
	$$
	\frac{k-1}{2} \le \lfloor\frac{k}{2}\rfloor\le\lceil\sum_{i\in I}s(i)\rceil-1
	$$
	$$
	k \le 2 \cdot \lceil\sum_{i\in I}s(i)\rceil-1\le 2\cdot k^\star -1
	$$
\end{solution}

\begin{problem}{6}
Given an undirected complete graph, each edge is assigned with a nonnegative cost by the function $c$. Find a Hamilton cycle with the largest cost by the greedy approach, and prove the guarantee factor
is 2.

\end{problem} 
\begin{solution}
	
Find the max spanning tree $T$ of graph $G$. By doubling its edges we obtain
an Eulerian graph connecting all vertices. Find an Euler tour of this graph, for example by traversing the edges in DFS (depth first search) order. Next obtain a Rudrata cycle by traversing the Euler tour and short-cutting vertices and previously visited vertices
\end{solution}
\end{document}